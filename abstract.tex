%!TEX encoding = UTF-8 Unicode

Hooded Merganser and Wood Duck are ducks that reproduce by building nests in tree cavities. Cavities can be supplemented with artificial nest boxes to improve management success. Nest boxes should be placed in different habitats that facilitate feeding and reproduction of each species.  Hooded Merganser dive below the water to pursue fish and thus should prefer deeper water. Wood Ducks feed on the surface or dabble and thus should prefer shallower water. To test this hypothesis, 10 nest boxes were randomly selected from each of 3 pools at Duck Creek Conservation Area near Puxico, Missouri. For each nest box, water depth and width, and tree coverage were measured. Each nest box was monitored weekly to determine nesting species and number of eggs hatched. Dump nesting was also assessed. 

Water depth and tree coverage were significantly different for Pool 1 compared to the other two pools. Nests occurred in 17 of 30 nest boxes. Sequential nests occurred in 6 of these boxes for 23 nest events total. Hooded Merganser presence and success were generally associated with the deeper pools. Wood Duck presence was associated with the shallower pools, but this did not appear to affect nesting success. This suggests that water depth may influence nest box selection by these two species. Thus, consideration of water depth adjacent to nest boxes may be an important consideration for waterfowl biologists and conservation managers.  