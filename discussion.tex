%!TEX encoding = UTF-8 Unicode

Duck Creek Conservation Area includes three large pools as part of their waterfowl management program. Pool~1 includes a large reservoir that is mostly open but includes some bottomland hardwood forest at the northern end. Nest boxes were located along two long drainage ditches bordering the large reservoir. The ditch on the west side of Pool~1 is generally wider and deeper than that on the east side (Figure~\ref{fig:water_width_plot}) because this segment is located upstream of a major water control structure used to manage water levels throughout \textsc{dcca}. Overall, water depth adjacent to nest boxes of Pool~1 was significantly greater than the other pools. The north end of Pool~1 is also adjacent to Unit~A, a large open tract of wet meadows and small pools. The open areas of Pool~1 and Unit~A, combined with the wider west ditch, created significantly less tree cover than the other pools (Table~\ref{tab:anova_coverage}; Figure~\ref{fig:tree_coverage_plot}). Pool~2 is divided between some open water (flooded marsh and agricultural field) and a large tract of bottomland wood, with boxes located inside both habitats.  This pool tends to dry out due to water management practices for the area but the nest boxes are located along ditches or low spots in the woods that tend to retain water (Figures~\ref{fig:nest_box_map} and~\ref{fig:water_depth_plot}).  Pool~3 is almost entirely bottomland woods, with greater variation of water depth (less so for width) across the pool. Nest boxes in this pool are located adjacent to multiple flooded ditches of various widths, wide flooded woods, and a small pond. Despite this variability, mean water width and depth were similar between Pools~2 and~3 (Figures~\ref{fig:water_width_plot} and~\ref{fig:water_depth_plot}) as was tree coverage (Figure~\ref{fig:tree_coverage_plot}). 

Habitat variation across pools may influence nest box choice of Hooded Merganser and Wood Duck. Most Hooded Merganser nests were in Pool~1 (3 nests) or Pool~3 (5 nests). This species used only one nest box in Pool~2. The lone Hooded Merganser nest in Pool~2 was in an isolated box located on a slightly deeper and wider depression, separate from the ditch where most boxes were located. Hooded Merganser had their greatest hatching success in Pool~1, where all 3 nests were located on the deeper west ditch. Twenty original eggs plus 2 dumped eggs were laid in nest box N93 and 18 of themhatched.  Sixteen original eggs were laid in N2 and all hatched (Table~\ref{tab:hooded_merganser_data_table}).  Most often, Hooded Merganser hatched between 8–11 eggs. Two nesting attempts in the South section failed, both in box S34. Interestingly, 9 eggs were laid for each attempt, but the nest was apparently abandoned each time. Overall Hooded Merganser nested in boxes adjacent to water from 0.6–4.5 m deep, with those in Pool~1 being the deepest. Hooded Merganser may prefer nest boxes near deeper water because their diving and sight-pursuit foraging strategy requires water deep enough to allow diving and swimming for larger macroinvertebrates (Salyer and Lagler 1940, K. M. Dugger et~al.~1999).  

Wood Ducks nested most often in Pool~2 (9 nests) or Pool~3 (4 nests) but their mean hatching success was higher in Pool~3 ($\overline{x} = 7.1~\mathrm{vs.} 13.5$; Table~\ref{tab:wood_duck_data_table}).  A single Wood Duck nested on the east side of Pool~1 (box N70) adjacent to boxes in Pool~2 (Figure~\ref{fig:nest_box_map}). The ditch on the north side of Pool~2 flows into the ditch on the east side of Pool~1 at approximately the location of N70. Thus, the habitat of N70 is more like Pool~2 boxes than boxes on the west side of Pool~1. Overall, Wood Duck nests were adjacent to water 0.15–1.5 m deep.   



Wood Ducks may prefer nest boxes near shallower water due to their foraging habit. Wood Ducks either surface feed or dabble (Dugger and Fredrickson 1992) and therefore can only reach food sources as far as they can reach from the surface (Foth et~al.~2014). Although Wood Ducks of both sexes consume emergent and floating vegetation and seeds (Landers et~al.~1977) but invertebrates compose most of the breeding female’s diet to provide protein during egg production (Drobney and Fredrickson 1979). These food sources would be found either emergent in shallow water or along the edge of the shore, regardless of maximum water depth. Although Pools 2 and 3 were similar in terms of width and depth of the ditches adjacent to the nest boxes, as well as coverage, Pool~3 had far greater surface water area (personal observation) that could be used by Wood Ducks for foraging. Flooded forests may average higher invertebrate biomass compared to other wetland habitats (Hagy and Kaminski 2012 , Foth et~al.~2014). The comparatively higher amount of flooded forest in Pool~3 therefore may provide better foraging for breeding Wood Duck hens, resulting in higher success.  

Dump nesting did not appear to affect the hatch rate of originally laid eggs. Hooded Merganser nests saw varying success with 4 of the 9 nests hatching all eggs laid. Two other nests, both in S34, failed with none of their eggs hatched. After the original nest was abandoned, a new hen added additional nesting material to cover up the original clutch, and then laid her own eggs. As the new hen covered the previous eggs, this was considered a new nesting event instead of dump nesting. The three remaining nests hatched most of their eggs at 91.7\% (S7), 84.6\% (S44), and 81.8\% (N93) hatched respectively. These three were the only Hooded Merganser nests that saw dump nesting during this study (Figure~\ref{fig:dumped_eggs_plot}). In most cases, the dumped eggs were found abandoned, likely due to eggs being dumped after incubation and development of the other eggs had begun. Only S7 saw heterospecific dump nesting with 2 Hooded Merganser eggs and 10 Wood Duck eggs being dumped. A Hooded Merganser hen was observed leaving the nest box before most inspection, suggesting that she was the original hen.  

Wood Ducks in contrast saw only 1 of 14 nests hatch all eggs. Three nests, E4, E22 attempt 2, and E93 attempt 2, failed with no eggs hatched. The rest of the nests had variable hatch rates ranging between 33.4–92.3\%, but most hatch rates fell between 60–80\%. Similar to the Hooded Merganser nests all Wood Duck nests with a partial hatch rate were observed to have been dump nested (Figure~\ref{fig:dumped_eggs_plot}). The nest in box E65 experienced heterospecific dump nesting where all other instances were intraspecific. Similar to Hooded Merganser the abandoned eggs were generally those that were dumped. Prior studies have suggested that dump nesting can have negative, minimal, or positive effects on nest hatch rates and overall duckling recruitment (Clawson et~al.~1979, Eadie et~al.~1988, Bakner et~al.~2024). For example, conspecific dump nesting by waterfowl might be advantageous in some cases when driven by low resource availability or K-type life history (Eadie et~al.~1988).  This study did not address resource availability but the results do suggest that dump nesting is not always a successful reproductive strategy for the dumping hen, as most dumped eggs were left abandoned. 



\section*{Management implications}

Nest box placement relative to water proximity, entrance orientation, and box heights have been historically studied for Wood Ducks, as has natural cavity use and food selection during breeding (Bellrose et~al.~1964, Drobney and Fredrickson 1979, Lacki et~al.~1987, Ryan et~al.~1998). Nest box use by Hooded Mergansers has been less studied (Zicus 1990, Heusmann and Stolarski 2017), hindering consistent management of nest boxes when both species are present. When Hooded Mergansers are present, providing nest boxes adjacent to deeper water might enhance nest box use by this species. Wood Ducks were more likely to use nest boxes adjacent to shallower water. This difference gives wildlife managers at places such as DCCA another variable to consider that might support populations of both species directly and effectively. Further, most ditches at DCCA are shallower than the Pool~1 west ditch. Increasing the number or length of deeper ditches could provide more locations to increase the number of nest boxes that appeal to Hooded Merganser. The deeper ditches must also provide a suitable foraging base for this diving species. Further, this apparent preference for different water depths may also allow for management strategies to be applied across larger public and private areas. This may include one-time management strategies such as altering topography to increase the depth of ditches where nest boxes may be placed. Then, annually variable strategies such as installing water control structures may allow for yearly alterations in water characteristics and increase depth variation to support both Hooded Merganser and Wood Duck.  

Water depth may be an important habitat variable to consider for nest box placement, but these results must be interpreted cautiously. Only 17 boxes of 30 selected were used by Hooded Merganser and Wood Duck. Repeated nest attempts increased the total nesting events to 23 but only for a single nesting season. The small number of nest boxes studied may have resulted in incomplete knowledge of nest box use by each species. For example, a total of 13 nest boxes are located along the wide ditch on the west side of the North section (MDC, unpublished data). Six of these were selected at random for this study but only 3 were used by Hooded Merganser (and none by Wood Duck). Hooded Merganser had their highest nesting success in 2 of these boxes. However, it is unknown if additional nests were in any of the 7 other boxes along the west ditch of Pool~1 during this study. If additional nests were present in any of these boxes, the data here may represent a biased view of nest box use or hatching success. 

A more extensive data set may better reveal contributions for the variables measured here. For example, 9 Wood Duck nests and 1 Hooded Merganser nest were found across Pool~2. While Pool~2 was a mixture of open and wooded areas 33 of 43 boxes were in areas greatly to fully surrounded by cover. Of the 10 boxes included in this study, 8 were located in the heavily wooded areas. In contrast, the other 2 boxes were in more open areas and did not house nests during the study but the data but whether this can be explained by coverage, another variable, or randomness cannot be determined. Finally, the results obtained here might be explained by other variables not measured, such as box height or distance to nearest water (Lacki et~al.~1987). 

Dump nesting should also be considered as part of a comprehensive management program. In many cases, such as dump nesting in Wood Duck nests E62 attempt 1, E22 attempt 1, and S20, the dumped eggs were the eggs that did not hatch. In other cases, such as S20 attempt 2, and S14, the total number of hatched eggs was less than the number of eggs laid by the original hen. Reduced hatching success cannot be directly attributed to dump nesting but these data suggest that this behavior can reduce hatch success of both species. In addition, the actual rate of dump nesting could be underestimated. Fewer eggs could be candled in Wood Duck nests due to federal permit restrictions. In both species, conspecific eggs could only be determined by lagging development in candling or by eggs being abandoned. Eggs dumped prior to the start of incubation by a conspecific individual would develop at the same rate as non-dumped eggs so could not be distinguished. Regardless of these issues, the presence of Hooded Merganser and Wood Duck together with many available nest boxes provides an opportunity to study this behavior in greater detail.  

Nest boxes are an artificial alternative to natural tree cavities used by Hooded Merganser and Wood Duck. Variables in naturally occurring tree cavities such as tree species, cavity height, and cavity depth may be compared to further understand the natural reproductive behaviors and success of these species (Ryan et~al.~1998). Nest success can also be compared between natural tree cavities and artificial nest boxes to better understand how cavity and box use may influence nesting success (Bellrose et~al.~1964).  

Finally, other study methods could be used to compare new measurements, such as stress of the nesting hen, against nesting success. Corticosterone has been used as a measure of physiological stress in birds (Bortolotti et~al.~2008, Romero and Fairhurst 2016, Johns et~al.~2018). Blood corticosterone has recently been used to measure handling stress in captive Wood Ducks (Broadus et~al.~2025). Thus, blood corticosterone levels might be useful to assess stress levels in wild ducks like Wood Ducks and Hooded Mergansers.  

